%% LyX 1.2 created this file.  For more info, see http://www.lyx.org/.
%% Do not edit unless you really know what you are doing.
\documentclass[english]{article}
\usepackage{pslatex}
\usepackage[T1]{fontenc}
\usepackage[latin1]{inputenc}
\usepackage{a4wide}

\makeatletter

%%%%%%%%%%%%%%%%%%%%%%%%%%%%%% LyX specific LaTeX commands.
\providecommand{\LyX}{L\kern-.1667em\lower.25em\hbox{Y}\kern-.125emX\@}

%%%%%%%%%%%%%%%%%%%%%%%%%%%%%% Textclass specific LaTeX commands.
 \newenvironment{lyxlist}[1]
   {\begin{list}{}
     {\settowidth{\labelwidth}{#1}
      \setlength{\leftmargin}{\labelwidth}
      \addtolength{\leftmargin}{\labelsep}
      \renewcommand{\makelabel}[1]{##1\hfil}}}
   {\end{list}}
 \newenvironment{lyxcode}
   {\begin{list}{}{
     \setlength{\rightmargin}{\leftmargin}
     \raggedright
     \setlength{\itemsep}{0pt}
     \setlength{\parsep}{0pt}
     \normalfont\ttfamily}%
    \item[]}
   {\end{list}}

\usepackage{babel}
\makeatother
\begin{document}

\title{emufxtool}


\author{Author: Nicola Orru' <nigu@itadinanta.it>}

\maketitle
\begin{abstract}
This project aims at supplying a sort of emu10k1-swiss-army-knife
for the ALSA project. The core of emufxtool is an assemble/unassemble
engine that can convert the program and data contained in the DSP
memory in and an human-readable form and vice-versa.

The emu10k1 DSP chip is the core of the creative SB Live!, Audigy,
PCI 512 and other sound cards.
\end{abstract}
\begin{description}
\item [Version:]0.2
\item [Parent-Project:]ALSA
\item [Based-on:]as10k1 A0.99
\item [Keywords:]Linux, Sound, ALSA, Creative, emu10k1, fx8010, SoundBlaster,
assembler, effects
\item [Original-Author:]Daniel Bertrand <d.bertrand@ieee.ca>, Jaroslav
Kysela <perex@suse.cz>.
\end{description}

\section{Project History }

I know I ought not start this document from the history of a project
that has no history, yet. You may ask why. Or you may not. However,
you'll find an answer at the end of this section. 

This project got life the minute I felt the need for chorus and reverb%
\footnote{By the way, the tool still CAN'T enable chorus and midi reverb : (%
} to be enabled in my SoundBlaster Live! WaveTable midi synthesiser,
just like the Creative Driver does under Windows 2000, so I subscribed
immediately to alsa-devel and asked the list about what to do to join
in.

Someone (maybe Takashi Iwai or Jaroslav Kysela, I don't remember -Yes,
I know I should peek at the archives, but let me tell the rest of
the story-) wrote me in response that I (or anyone else) had to sew
a patch to the kernel driver in order to enable the FX\_BUSes for
chorus and reverb, and to write the loader code, while an assembler
program that appeared to be suitable for ALSA had already been provided
by \textbf{Daniel Bertrand} and \textbf{Jaroslav Kysela} (thanks!).

This is the way my odyssey started: I downloaded all the docs I was
able to find about the DSP, the assembler code, the driver source,
then I began to try and experiment with them... just to discover that
a loader based on an OSS-oriented assembler couldn't work, first of
all because of the completely different patch management mechanisms
and effect routing inside the drivers. So I had to start again. Fortunately,
most of the code that I wrote to perform microcode unassembly was
useful, thus it was immediately embedded in the new version of the
tool.

Trying to expand my horizons, I moved the project's objective, from
a simple loader to the aforementioned \char`\"{}swiss knife\char`\"{}
for the fx8010 DSP processor. What was originally thought as \char`\"{}ld10k1\char`\"{}
(loader for emu10k1) became \char`\"{}emufxtool\char`\"{}. 

That's where I am: I managed to implement a beta version of a tool
that can

\begin{itemize}
\item assemble DSP code, featuring a powerful one-pass macro preprocessor,
an expression parser/evaluator, and typed variable declarations
\item unassemble contents of the DSP registers and code in an human-readable
format, that can be easily modified, recompiled and reloaded.
\item unassemble compiled DSP patches 
\item load compiled DSP patches into the emu10k1 iron (with a relocation
mechanism, designed but still to be implemented)
\end{itemize}
and, meantime, there's where I got stuck. My comprehension of how
the fx8010 (the core of the emu10k1) works is far from complete, neither
my knowledge of ALSA internals is. I have to go and study some more
(and more), assuming I will be able to obtain more documentation about
the chip.

Now you have the answer to the initial question: I need help. If you
reached this very word, maybe you're interested in giving me a hand.
If not... Well, who am I supposed to be speaking to? : )

Anyway, what I need the most is feedback from experts in the ALSA
architecture and, particularly, in the emu10k1 driver. Feedback from
testers and users is also vital.


\section{Installation}


\subsection{Requirements}

What you do need are:

\begin{itemize}
\item GCC 3.2
\item the C++ standard template library (installed in /usr/include/c++/3.2
\item ALSA 0.9.0rc7 (or later?)
\item GNU make
\end{itemize}
Soon (or later) I will provide an autoconf/automake makefile generator
and installer. Once I learn autoconf and automake...


\subsection{Building}

You can make and run this program by the canonical sequence:

\begin{lyxcode}
tar~-xvzf~emufxtool-version.number.tar.gz

cd~emufxtool-version.number

make

make~install~(as~root)

make~test~(as~root)
\end{lyxcode}
If make terminates successfully, you will end up with a

\begin{lyxcode}
emufxtool
\end{lyxcode}
binary executable in /usr/local/bin.


\section{Usage}


\subsection{Invocation}

\begin{lyxcode}
emufxtool~\textbf{action}~{[}-o~\textbf{asmoptions}{]}~{[}-d~\textbf{device},~defaults~to~hw:0,0{]}~\textbackslash{}

{[}-y~\textbf{additional\_sYmbol\_table}{]}~{[}\textbf{dsp~patch~files...}{]}
\end{lyxcode}

\paragraph{action}

\begin{lyxlist}{00.00.0000}
\item ['-l']load patches into emu hardware
\item ['-t']test patches sanity, useful with -oa, see later
\item ['-a']assemble patches
\item ['-h']help
\item ['-v']version
\item ['-u']unassemble patch found in emu hardware
\item ['-s']save hardware contents in a patch file
\end{lyxlist}

\paragraph{asmoptions }

(multiple options allowed, latest override earliest):

\begin{lyxlist}{00.00.0000}
\item ['a']all options on (default all off)
\item ['v']enables 
\item ['V']disables patch dump to stdout
\item ['t']enables 
\item ['T']disables token analysis dump to stdout
\item ['p']enables 
\item ['P']disables parsing only (no files generated during compilation)
\item ['m']enables 
\item ['M']disables macro expansion to stdout
\end{lyxlist}

\paragraph{Notes}

\begin{itemize}
\item Every compilation process generates a \char`\"{}patch\_name.emufx\char`\"{}
file for each assembled patch, overriding any previously existing
file named that way.
\item Every compilation process generates an additional \char`\"{}patch\_name.sym\char`\"{}
containing the symbols, that can be used later to unassemble the same
patch or the DSP contents properly, overriding existing files.
\end{itemize}

\subsubsection{Examples}

\begin{lyxcode}
emufxtool~-a~default.as10k1
\end{lyxcode}
compiles \char`\"{}default.as10k1\char`\"{}, creating the binary patch
\char`\"{}default.emufx\char`\"{}

\begin{lyxcode}
emufxtool~-a~-oa~{*}.as10k1~
\end{lyxcode}
compiles any patches found in the current path, creating a {*}.emufx
for each file, dumping a very verbose list of messages to stdout

\begin{lyxcode}
emufxtool~-t~-ov~default.emufx~-y~reverse.sym
\end{lyxcode}
dumps all the contents of the binary patch default.emufx to stdout,
using the symbols contained in reverse.sym

\begin{lyxcode}
emufxtool~-l~-d~hw:0,0~default.emufx
\end{lyxcode}
loads default.emufx into the DSP memory and registers through the
hw:0,0 interface

\begin{lyxcode}
emufxtool~-h
\end{lyxcode}
help

\begin{lyxcode}
emufxtool~-u~-ov
\end{lyxcode}
dumps the contents of the DSP memory onto stdout

\begin{lyxcode}
emufxtool~-s~dump.emufx
\end{lyxcode}
dumps the contents of the DSP memory to a binary patch file called
dump.emufx


\section{Language Syntax }


\subsection{Meta-Language}

The language structure is described by the following BNF-like rules,
where

\begin{lyxlist}{00.00.0000}
\item [\char`\"{}string\char`\"{}]is a fixed literal string
\item ['expr']is a regular expression
\item [\{\}]means zero or more items,
\item [{[}{]}]means zero or one item
\item [::=]means \char`\"{}means\char`\"{}
\item [<null>]represent an empty string, i.e. nothing
\item [bare-words]represent language elements
\end{lyxlist}

\subsection{Basic Rules}

The language is \char`\"{}free form\char`\"{}, meaning you can insert
any number of spaces, tabs and newlines between elements as you prefer
(except inside strings).

The assembler applies no limits to line length or word length. All
identifiers and keywords are case sensitive. The syntax was chosen
with an eye to the preservation of compatibility with the original
as10k1 project, in order to ease the porting of existing OSS patches
to ALSA.


\subsubsection{Patch structure}

A so called \char`\"{}assembly patch\char`\"{} must be written in
an ASCII text file. In a patch, you can mix, as you like, instructions
and declarations. You can use PREPROCESSOR DIRECTIVES whenever you
like, for both instructions and declarations. Preprocessor directives
will be described aside, as they don't represent language statements.

\begin{lyxlist}{00.00.0000}
\item [patch]::= \{element {[}\char`\"{};\char`\"{}{]}\}
\item [element]::= instruction | declaration | <null>
\end{lyxlist}

\subsubsection{instructions}

All instructions require 4 operands. The format is:

\begin{lyxlist}{00.00.0000}
\item [instruction]::= {[}label{]} opcode expression \char`\"{},\char`\"{}
expression \char`\"{},\char`\"{} expression \char`\"{},\char`\"{}
expression {[}\char`\"{};\char`\"{}{]} | {[}label{]} <null>
\item [label]::= identifier\char`\"{}:\char`\"{}
\item [opcode]::= \char`\"{}MACS\char`\"{} | \char`\"{}MACS1\char`\"{}
| \char`\"{}MACW\char`\"{} | \char`\"{}MACW1\char`\"{} | \char`\"{}MACINTS\char`\"{}
| \char`\"{}MACINTW\char`\"{} | \char`\"{}ACC3\char`\"{} | \char`\"{}MACMV\char`\"{}
| \char`\"{}ANDXOR\char`\"{} | \char`\"{}TSTNEG\char`\"{} | \char`\"{}LIMIT\char`\"{}
| \char`\"{}LIMIT1\char`\"{} | \char`\"{}LOG\char`\"{} | \char`\"{}EXP\char`\"{}
| \char`\"{}INTERP\char`\"{} | \char`\"{}MAC0\char`\"{} | \char`\"{}MAC1\char`\"{}
| \char`\"{}MAC2\char`\"{} | \char`\"{}MAC3\char`\"{} | \char`\"{}MACNS\char`\"{}
| \char`\"{}MACNW\char`\"{} | \char`\"{}MACINT\char`\"{} | \char`\"{}MACINTNW\char`\"{}
| \char`\"{}LIMIT0\char`\"{} | \char`\"{}LIMITGE\char`\"{} | \char`\"{}LIMITL\char`\"{}
\end{lyxlist}
where expressions mean operands in the following order: R, A, X, Y.
Note that some documentation out there call the R operand Z and the
A operand W.

Later in this document you can see what an \char`\"{}expression\char`\"{}
is.


\paragraph{Opcodes}

Here are 16 opcodes.

\begin{lyxlist}{00.00.0000}
\item [MACS:]R = A%
\footnote{Special note on the accumulator: MAC{*} instructions with ACCUM as
A operand use Most significant 32 bits. MACINT{*} instructions with
ACCUM as A operand use Least significant 32 bits.%
} + (X {*} Y >\,{}> 31) ; saturation
\item [MACS1:]R = A + (-X {*} Y >\,{}> 31) ; saturation
\item [MACW:]R = A + (X {*} Y >\,{}> 31) ; wraparound
\item [MACW1:]R = A + (-X {*} Y >\,{}> 31) ; wraparound
\item [MACINTS:]R = A + X {*} Y ; saturation
\item [MACINTW:]R = A + X {*} Y ; wraparound (31-bit)
\item [ACC3:]R = A + X + Y ; saturation
\item [MACMV:]R = A, acc += X {*} Y >\,{}> 31
\item [ANDXOR:]R = (A \& X) \textasciicircum{} Y
\item [TSTNEG:]R = (A >= Y) ? X : \textasciitilde{}X
\item [LIMIT:]R = (A >= Y) ? X : Y
\item [LIMIT1:]R = (A < Y) ? X : Y
\item [LOG:]... /{*} FIXME: how does it work? {*}/
\item [EXP:]... /{*} FIXME: how does it work? {*}/
\item [INTERP:]R = A + (X {*} (Y - A) >\,{}> 31) ; saturation
\item [SKIP:]R,CCR,CC\_TEST,COUNT 
\end{lyxlist}
several of these have predefined aliases:

\begin{lyxlist}{00.00.0000}
\item [MACS]is aliased by MAC0
\item [MACS1]is aliased by MAC1 or MACNS
\item [MACW]is aliased by MAC2
\item [MACW1]is aliased by MAC3 or MACNW
\item [MACINTS]is aliased by MACINT or MACINT0
\item [MACINTW]is aliased by MACINT1
\item [LIMIT]is aliased by LIMIT0 or LIMITGE
\item [LIMIT1]is aliased by LIMITL
\end{lyxlist}
For more details on the emu10k1 see the dsp.txt file distributed with
the linux driver.

You may optionally terminate an instruction by a \char`\"{};\char`\"{}.
You can chain more than one instruction on the same line or subdivide
one instruction into many lines.


\paragraph{Operands}

are always specified as expressions. The assembler evaluates the expression,
then compiles the result into an address of a register (as the operands
for emu10k1 instructions are always addresses of registers), according
to the expression operands type.

\begin{lyxlist}{00.00.0000}
\item [expression]::= \char`\"{}(\char`\"{} expression \char`\"{})\char`\"{}
| expression operator expression | literal\_constant | identifier
| unary\_operator expression | array\_element | label\_reference
\item [operator]::= \char`\"{}+\char`\"{} | \char`\"{}-\char`\"{} | \char`\"{}\textasciicircum{}\char`\"{}
| \char`\"{}<\,{}<\char`\"{} | \char`\"{}>\,{}>\char`\"{}
| \char`\"{}==\char`\"{} | \char`\"{}>\char`\"{} | \char`\"{}<\char`\"{}
| \char`\"{}<=\char`\"{} | \char`\"{}>=\char`\"{} | \char`\"{}|\char`\"{}
| \char`\"{}\&\char`\"{} | \char`\"{}||\char`\"{} | \char`\"{}\&\&\char`\"{}
| \char`\"{}{*}\char`\"{} | \char`\"{}\%\char`\"{} | \char`\"{}/\char`\"{}
\item [unary\_operator]::= \textasciitilde{} | ! | - | + 
\item [identifier]::= '{[}A-Za-z{]}{[}A-Za-z0-9\textbackslash{}.{]}+'
\item [array\_element]::= identifier \char`\"{}{[}\char`\"{} expression
\char`\"{}{]}\char`\"{}
\item [literal\_constant]::= decimal\_constant | hex\_constants | binary\_constant
| 
\item [floating-point\_constant]| octal\_constant | time\_constant
\item [label\_reference]::= \char`\"{}:\char`\"{} identifier
\item [decimal\_constant]::= '{[}0-9{]}+'
\item [hex\_constant]::= '0x{[}0-9a-fA-F{]}+' | '\textbackslash{}\${[}0-9a-fA-F{]}+'
\item [octal\_constant]::= '@{[}0-7{]}+'
\item [binary\_constant]::= '\%{[}01{]}+'
\item [floating-point\_constant]::= '\#?'decimal\_constant'(\textbackslash{}.decimal\_constant)?''(E{[}+-{]}decimal\_constant)?'
\item [time\_constant]::= \char`\"{}\&\char`\"{}decimal\_constant'(\textbackslash{}.decimal\_constant)?''(E{[}+-{]}decimal\_constant)?'
\end{lyxlist}

\paragraph{Notes}

\begin{itemize}
\item Floating point values are always (and immediately) converted to fixed
point numbers.
\item Time constants are expressed in seconds and immediately converted
to samples (integer). So, for instance, \&1.0 (one second) is converted
to 48000; \#1.0 is converted to 0x00010000.
\item All arithmetic is performed in fixed point/integer space. If you are
looking for examples, look at \char`\"{}default.as10k1\char`\"{}
\end{itemize}

\subsubsection{Declarations}

you can declare variables, symbolic constants, controls, TRAM lines,
and TRAMs


\paragraph{Initialisers}

An initialiser is either a constant, a variable, a control, a TRAM
or a line that is bound to an unique name. To declare one or more
variables you should use the following syntax (refer to the expression
syntax):

\begin{lyxlist}{00.00.0000}
\item [declaration]::= var\_declaration | const\_declaration | control\_declaration
| tram\_declaration | tram\_line\_declaration \char`\"{};\char`\"{}
\item [var\_declaration]::= \char`\"{}var\char`\"{} {[}\char`\"{}absolute\char`\"{}{]}
{[}\char`\"{}export\char`\"{}{]} {[}var\_type{]} var\_initialiser
\{, var\_initialiser\}
\item [const\_declaration]::= \char`\"{}const\char`\"{} {[}\char`\"{}export\char`\"{}{]}
var\_initialiser \{, var\_initialiser\}
\item [control\_declaration]::= \char`\"{}control\char`\"{} {[}control\_type{]}
{[}control\_value{]} atom\_initialiser \{,atom\_initialiser\}
\item [tram\_line\_declaration]::= \char`\"{}line\char`\"{} atom\_initialiser
\{,atom\_initialiser\}
\item [tram\_declaration]::= \char`\"{}tram\char`\"{} identifier attribute\_list
\item [control\_type]::= <null> | \char`\"{}mono\char`\"{} | \char`\"{}stereo\char`\"{}
\item [control\_value]::= <null> | \char`\"{}onoff\char`\"{} | \char`\"{}range\char`\"{}
\item [var\_initialiser]::= atom\_initialiser | array\_initialiser
\item [var\_type]::= <null> |\char`\"{}register\char`\"{} | \char`\"{}fxbus\char`\"{}
| \char`\"{}extin\char`\"{} | \char`\"{}extout\char`\"{}
\item [atom\_initialiser]::= identifier {[} \char`\"{}=>\char`\"{} hex\_constant
{]} {[} \char`\"{}=\char`\"{} expression {]} {[} attribute\_list {]};
\item [array\_initialiser]::= identifier {[} \char`\"{}{[}\char`\"{} expression
\char`\"{}{]}\char`\"{} {]} {[} \char`\"{}=>\char`\"{} hex\_constant
{]} {[} \char`\"{}=\char`\"{} expression\_list {]}
\item [expression\_list]::= \char`\"{}\{\char`\"{} expression \{,expression\}
\char`\"{}\}\char`\"{}
\item [attribute\_list]::= \char`\"{}\{\char`\"{} {[}attribute\_definition{]}
{[},attribute\_definition{]} \char`\"{}\}\char`\"{}
\item [attribute\_definition]::= identifier \char`\"{}=\char`\"{} expression
| identifier \char`\"{}=\char`\"{} string
\item [string]::= \char`\"{}\textbackslash{}\char`\"{}\char`\"{} '{[}\textasciicircum{}\textbackslash{}\char`\"{}{]}{*}'
\char`\"{}\textbackslash{}\char`\"{}\char`\"{}
\end{lyxlist}

\paragraph{Semantic}

\begin{description}
\item [const]When a CONSTANT is declared, using \char`\"{}const\char`\"{},
it is inserted in a temporary symbol table, and its value used in
expression. A GPR is allocated only if necessary; multiple constants
can share the same GPR or use an hardware constant.
\item [var]When a VAR is declared, by means of \char`\"{}var\char`\"{},
a GPR is allocated for each element if the var is not assigned a GPR
directly. If a var is givenone or more values, they can be used as
constants during assembly times. Arrays are always allocated contiguously.
GPRs are assigned in declaration order (earlier declared vars are
given the lowest addresses), starting from the first GPR available
(usually 0x100 for the Live!). Vars can be declared \char`\"{}absolute\char`\"{}
and thus an address can be assigned to the variable. This way, no
GPRs are allocated for the variable. An example follows:

\begin{lyxcode}
var~absolute~VARIABLE\_NAME~=>~0x1ee;~
\end{lyxcode}
Vars and constants always have empty attribute lists.

\item [control]When a \char`\"{}control\char`\"{} is declared, it can be
either MONO or STEREO. Mono is a single value, whilst stereo means
a vector (array) of two distinct values, to be used for left and right
channels. This means a MONO control uses a GPR, and stereo control
uses two GPRs. Valid attributes for controls in an attribute list
are:

\begin{itemize}
\item \char`\"{}name\char`\"{} (string): contains the name of the control
\item \char`\"{}min\char`\"{} and \char`\"{}max\char`\"{} (numeric): define
the control's range
\item \char`\"{}index\char`\"{} (numeric): assigns an internal index to
the control (FIXME: let me understand this first!)
\item \char`\"{}translation\char`\"{} (string): can be \char`\"{}t100\char`\"{}
(default for \char`\"{}range\char`\"{} controls), 
\item \char`\"{}bass\char`\"{}, \char`\"{}treble\char`\"{} or \char`\"{}onoff\char`\"{}
(default for \char`\"{}onoff\char`\"{} controls). Sets the translation
table inside the driver (FIXME: understand this, too...) 
\end{itemize}
when a \char`\"{}stereo\char`\"{} control is declared, for instance
as VOLUME, three variables are actually created:

\begin{lyxlist}{00.00.0000}
\item [\textbf{VOLUME},]that is an array of two elements, VOLUME{[}0{]}
and VOLUME{[}1{]}
\item [\textbf{VOLUME.left},]that aliases VOLUME{[}0{]}
\item [\textbf{VOLUME.right},]that aliases VOLUME{[}1{]}
\end{lyxlist}
The control initial value can be assigned by \char`\"{}=\char`\"{}
in the declaration

\item [tram]declarations define the amount of TRAM memory to allocate,
for being used by \char`\"{}line\char`\"{} declaration.
\item [line]A \char`\"{}line\char`\"{} is meant to be a \char`\"{}point\char`\"{}
where you can access tram memory, for reading and/or writing. The
emu10k1 DSP reserves a different address space to lines, which is
used to assign address to these variables. When you declare a \char`\"{}line\char`\"{}
initialiser, actually three variables are created:

\begin{lyxlist}{00.00.0000}
\item [\textbf{LINE},]that points to the first \char`\"{}data line register\char`\"{}
available
\item [\textbf{LINE.data}]that aliases \char`\"{}LINE\char`\"{}
\item [\textbf{LINE.address}]~
\end{lyxlist}
\end{description}

\paragraph{Examples}

\begin{lyxcode}
control~stereo~range~VOLUME~=~\{100,100\}~\{name=\char`\"{}Main~Volume\char`\"{},~min~=~0,~max~=~100\};
\end{lyxcode}
declares a GPR stereo control called \char`\"{}Main Volume\char`\"{},
with an initial value of 100, with range between 0 and 100

\begin{lyxcode}
control~stereo~onoff~VOLUME\_SWITCH~=~1~\{name=\char`\"{}Main~Switch\char`\"{},~index=0\};
\end{lyxcode}
declares a GPR switch-type volume called \char`\"{}Main Switch\char`\"{},
initially on


\subsection{Macro preprocessor}

The so-called preprocessor is not actually a preprocessor, as it works
in \char`\"{}real time\char`\"{} during compilation with no need for
a second-pass, but it performs the same way: you can write \char`\"{}parametric\char`\"{}
constructs that can \char`\"{}expand\char`\"{} to actual code, preventing
you from rewriting entire blocks that are alike. In a system like
the emu10k1 DSP, where you can't use calls or loops, a macro system
is the only way to emulate subroutines and structures.

But I guess that you already know what a preprocessor is, so let's
go on.

Macro preprocessing allows you to write comments in \char`\"{}c-style\char`\"{}
\textbf{/{*} {*}/}. When a comment is encountered, its contents are
unconditionally discarded. Nested comments are not allowed.

Macro blocks may contain everything, including other preprocessing
directives, that should expand correctly. This behaviour lets you
compose quite weird structures, like nested loops or even recursive
macros (beware of infinite recursion!).

The directives start with a dotted {}``.keyword'' and can be used
everywhere in the code, being possible to declare them everywhere
in the middle of a line, too.

\begin{lyxcode}
.include~\char`\"{}filename\char`\"{}
\end{lyxcode}
This directive substitutes itself with the contents of the included
\char`\"{}filename\char`\"{}

\begin{lyxcode}
.define~macro\_name(param1,~param2...)~\{~block~\}

.macro~macro\_name(param1,~param2...)~\{~block~\}
\end{lyxcode}
creates a macro called \char`\"{}macro name\char`\"{}. 

It substitutes itself with a null string, but whenever the defined
symbol is found in the code, from now on, it will be expanded to \char`\"{}block\char`\"{}.
Parameter substitution is performed inside the \char`\"{}block\char`\"{}.
That is, if the macro is invoked as

\begin{lyxcode}
macro\_name(a,b,...)
\end{lyxcode}
the invocation is substituted with \char`\"{}block\char`\"{}. Whenever
\char`\"{}param1\char`\"{} is found in the block (as a distinct token)
it is substituted by \char`\"{}a\char`\"{}, param2 is replaced by
\char`\"{}b\char`\"{} and so on. The number of arguments during invocation
must be equal to the number of parameters the macro has been declared
with.

Parameter-less macros can be declared (and invoked) as

\begin{lyxcode}
macro\_no\_params()
\end{lyxcode}
with a null list of parameters. Parentheses are always mandatory (but
this may change in the future). An alternative way to invoke a macro
and perform substitution is:

\begin{lyxcode}
macro\_name~a,~b...~;
\end{lyxcode}
In this condition, the final \char`\"{};\char`\"{} is mandatory, but
this syntax is deprecated (although included) because it may lead
to ambiguities.

\begin{lyxcode}
.if~expression~\{~block~\}
\end{lyxcode}
This directive evaluates the expression using literal constant and
initialisers defined before, and substitutes itself to \char`\"{}block\char`\"{}
if the expression is not zero.

\begin{lyxcode}
.if~expression~\{~block1~\}~else~\{~block2~\}
\end{lyxcode}
This directive evaluates the expression using literal constant and
initialisers defined before, and substitutes itself to \char`\"{}block1\char`\"{}
if the expression is not zero or \char`\"{}block2\char`\"{} if the
expression evaluates to zero.

\begin{lyxcode}
.eval~expression
\end{lyxcode}
evaluates \char`\"{}expression\char`\"{} and substitutes itself to
the result. Useful, in combination with \char`\"{}.define\char`\"{}
if you have very complex constant expression often used in the code
and you want to avoid re-evaluating them lots of times. 

\begin{lyxcode}
.defined~symbol
\end{lyxcode}
replaces itself with \char`\"{}1\char`\"{} if the symbol is a defined
initialiser or macro, or with \char`\"{}0\char`\"{} otherwise.

\begin{lyxcode}
.undefined~symbol
\end{lyxcode}
like the latter, with meaning inverted. 

\begin{lyxcode}
.for~identifier~=~start:end~\{~block~\}
\end{lyxcode}
start and end are expression that must evaluate to integer values
and express a sequence (ascending or descending) of numbers. For each
number in this sequence, the directive replaces itself with the \char`\"{}block\char`\"{},
replacing the token \char`\"{}identifier\char`\"{} inside the block
with the current value of the sequence number.

\begin{lyxcode}
.file
\end{lyxcode}
replaces itself with the name of the file currently being read

\begin{lyxcode}
.line
\end{lyxcode}
replaces itself with the line number currently being read

\begin{lyxcode}
.warn~\{~message~\}
\end{lyxcode}
replaces itself with nothing, expands \char`\"{}message\char`\"{}
using the current macro definition and outputs it to stderr.

\begin{lyxcode}
.err~\{~message~\}
\end{lyxcode}
behaves like .warn, but generates an user error that leads to a failure
of the compilation process. if a .err is encountered, the current
compilation does not create an output file.


\section{TODO}

\begin{description}
\item [code~style.]At this stage, code actually works, although not perfectly,
but not enough effort was spent to make the code readable, understandable
or coherent.
\item [documentation.]No comment. The only existing documentation is this
README. More documentation will be available as soon as possibile.
I just started experimenting with doxygen.
\item [TRAM~setup~support.]Support for internal and external TRAM must
be improved 
\item [PCM~setup~support.]PCM setup is not supported yet
\item [More~sound~cards.]The only card tested and \char`\"{}fully\char`\"{}
supported is the SoundBlaster Live! 256, as I haven't got other emu10k1
cards on my system. Other cards will be easily supported, as all card
parameters are \char`\"{}modularized\char`\"{} in a single class.
\item [automake/autoconf~build~system.]I am not acquainted enough with
autotools.
\item [cvs~support.]As this project was intended to be a part of a largest
one, I think the better choice I can take is delegating to ALSA maintainers
the decision about where to store its files. They may be merged into
the ALSA repository, so I guess it will be included in the \char`\"{}alsa-tools\char`\"{}
cvs, or it survive as a standalone project. In the latter case, an
autonomous repository will be set up.
\end{description}

\section{How to contribute}

You can help me in the development of this project by:

\begin{itemize}
\item Testing the program, submitting bug reports and wish lists
\item Submitting patches or fragments of code
\item Proofreading and reviewing code and documentation
\item Sending me corrections and advices that can help me to improve my
writing (english is not my first language)
\item Any Other Business
\end{itemize}

\subsection{Contact}

You can contact me (Nicola Orru') for any reason, by email (of course),
at the following addresses:

\begin{lyxlist}{00.00.0000}
\item [\textbf{nigu@itadinanta.it}](personal email)
\item [\textbf{no@energit.it}](office email)
\end{lyxlist}
If you write me about this project, the email subject should start
with the word \textbf{{[}emufxtool{]}}

\newpage
\tableofcontents{}
\end{document}
